\section{Rolverdeling}
Om de organisatie van de studiereis naar het
buitenland gestructureerd en efficiënt te laten ver
lopen, is er binnen de commissie een duideli
jke taakverdeling opgesteld. Elke functie heeft zijn
eigen verantwoordelijkheden, waardoor er overzicht
en duidelijkheid ontstaat. Onderstaand worden de
verschillende rollen en hun taken toegelicht.

\begin{itemize}
	\item \textbf{Voorzitter} De voorzitter is verantwoordelijk voor de al
	gemene cöordinatie van de commissie. Hij of zij
	bewaakt de voortgang van de voorbereidingen,
	zorgt dat deadlines worden gehaald en dat de
	communicatie tussen de commissieleden goed
	verloopt. Daarnaast vertegenwoordigt de voorzit
	ter de commissie naar buiten toe, bijvoorbeeld
	richting de opleiding en de begeleidende docenten.
	Ook moet de voorzitter aanwezig zijn bij
	alle voorzittersvergaderingen zijn. De voorzitter
	moet iemand zijn die goed overzicht kan houden,
	communicatief sterk is en makkelijk besluiten
	neemt.
	
	\vspace{0.5cm}
	
	\item \textbf{Penningmeester} De penningmeester houdt toezicht op alle financiële zaken rondom de reis. Dit betekent
	het opstellen en beheren van het budget, het
	bijhouden van inkomsten en uitgaven, en het zor
	gen voor een duidelijke financiële verantwoording.
	Ook beheert de penningmeester de betalingen van deelnemers.
	De penningmeester be
	houdt daarnaast nauw contact met de algemene penningmeester van de opleiding die uiteindelijk
	ook het budget van de buitenlandreis beschik
	baar stelt. De penningmeester moet iemand zijn
	die nauwkeurig werkt, goed met cijfers omgaat
	en financieel inzicht heeft.
	
	\item \textbf{Bedrijfscontact} Het commissielid verantwoordelijk voor bedrijf
	scontact onderhoudt de communicatie met de
	bedrijven die tijdens de reis worden bezocht.
	Deze persoon legt het eerste contact, plant af
	spraken in en bevestigt de bezoekmomenten.
	Ook zorgt hij of zij ervoor dat bedrijven tijdig alle
	benodigde informatie ontvangen en dat het programma inhoudelijk aansluit bij de doelen van de
	studiereis. Ook moet er samengewerkt worden
	met de rol logistiek aangezien er vervoer moet
	komen naar het bedrijf. Degene die bedrijfscontact op zich neemt moet iemand zijn die vlot kan
	communiceren, zakelijk correct is en niet bang is
	om bedrijven te benaderen.
	
	\vspace{0.5cm}
	
	\item \textbf{Cöordinator activiteiten \& logistiek} Het commissielid verantwoordelijk voor vervoer
	en activiteiten zorgt voor een goed georgan
	iseerde logistiek en een aantrekkelijk nevenpro
	gramma. Dit houdt in dat hij of zij alle trans
	portzaken regelt, zoals het boeken van de heen
	en terugreis en het lokale vervoer tussen verblijf,
	bedrijven en activiteiten. Deze persoon houdt
	ook nauw contact met de activiteitencommissie
	aangezien zij de activiteiten daadwerkelijk or
	ganiseren. Degene die deze rol op zich neemt
	moet iemand zijn die praktisch ingesteld is, goed
	kan plannen en creatief is in het bedenken van
	activiteiten.
\end{itemize}

Derolverdeling is gedaan op basis van persoonlijke
voorkeuren vooral op basis van wie welke rol het
leukst leek. De verdeling is als volgt opgedeeld:

\begin{table}[h!]
	\centering
	\caption{Takenverdeling commissie BK5 2025-2026}
	\label{tab:takenverdeling}
	\begin{tabular}{|l|l|}
		\hline
		\textbf{Naam} & \textbf{Taak} \\
		\hline
		Marijn van der Gracht & Voorzitter \\
		\hline
		Stijn van Straaten & Penningmeester \\
		\hline
		Tristan van Duuren & Coördinator activiteiten \& logistiek \\
		\hline
		Florent Kegler & Bedrijfscontact \\
		\hline
	\end{tabular}
\end{table}