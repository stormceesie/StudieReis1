\section{Risico's en Maatregelen}

Dit hoofdstuk behandelt de belangrijkste risico’s die tijdens de studiereis kunnen optreden en de maatregelen die genomen kunnen worden om deze te beperken. De risicoanalyse is ingedeeld per categorie, zoals vervoer, verblijf, gezondheid en veiligheid, communicatie en bedrijfsbezoeken. Door vooraf mogelijke problemen in kaart te brengen en preventieve acties te definiëren, kan de reis veilig, soepel en plezierig verlopen voor alle deelnemers en begeleiders.

\begin{table}[h!]
	\centering
	\caption{Risicoanalyse studiereis BK5 2025-2026}
	\label{tab:risicoanalyse}
	\begin{tabular}{|l|p{6cm}|p{5cm}|}
		\hline
		\textbf{Categorie} & \textbf{Risico} & \textbf{Maatregelen / Preventie} \\
		\hline
		Vervoer (vliegen) & Vluchtvertraging of annulering, verlies/beschadiging bagage, groepsleden missen vlucht & Tijdig boeken en bevestigen vluchten, tijdig op luchthaven aanwezig (2–3 uur voor vertrek), zorgen dat iedereen zijn/haar reisverzekering op orde heeft, contactgegevens en reisschema delen met deelnemers \\
		\hline
		Verblijf / accommodatie & Diefstal, ongewenst gedrag van gasten, ontevredenheid/onduidelijkheid over kamerindeling & Hostel/hotel met goede veiligheidsstandaarden kiezen, waardevolle spullen in kluis bewaren, gedragscode bespreken met groep, voorkeuren kamerindeling peilen en vooraf indeling maken \\
		\hline
		Lokaal vervoer & Ongevallen in openbaar vervoer/taxi, deelnemers verdwalen of komen te laat & Heldere afspraken over verzamelpunten/tijden, gebruik betrouwbare taxi’s of openbaar vervoer, telefoonnummers en WhatsApp-groep voor noodgevallen \\
		\hline
		Gezondheid en veiligheid & Ziekte tijdens reis, allergieën, ongevallen & Medische gegevens (bijv. allergieën) vooraf inventariseren, noodnummer(s) verzamelen en delen, reisverzekering inclusief medische dekking \\
		\hline
		Culturele en juridische verschillen & Overtreden lokale wetten, culturele misverstanden & Belangrijkste punten vermelden over lokale wet- en regelgeving, informeren over omgangsvormen en gebruiken \\
		\hline
		Communicatie en coördinatie & Onduidelijkheid in planning, calamiteiten en gedrag, personen vergeten & Centraal aanspreekpunt/reisleider per dag, WhatsApp-groep maken voor het delen van informatie, dagelijkse check-in momentjes (koppen tellen of alternatief), draaiboek maken met alle belangrijke reisinformatie op één plek \\
		\hline
		Bedrijfsbezoeken & Te laat komen, ongepaste kleding of gedrag, ontevredenheid over de bedrijfsbezoeken & Dresscode en gedragsregels vooraf communiceren, programma en adresgegevens vooraf delen, op tijd vertrekken met reismarge, korte briefing per bezoek over cultuur en etiquette, diverse bedrijfssoorten uitkiezen en voorkeuren peilen \\
		\hline
		Alcoholgebruik & Overmatig alcoholgebruik, onder invloed tijdens bedrijfsbezoeken & Vooraf duidelijke gedragsregels opstellen in lijn met de regels vanuit het bestuur, aanspreekpunt aanwijzen, alcoholgebruik vermijden vóór en tijdens zakelijke bezoeken \\
		\hline
	\end{tabular}
\end{table}