\section{Evaluatie en Verantwoording}

De evaluatie en verantwoording van de internationale studiereis zijn essentieel om inzicht te krijgen in de mate waarin de vooraf gestelde doelstellingen zijn gerealiseerd en om te leren voor toekomstige studiereizen. Dit hoofdstuk beschrijft de methoden en instrumenten waarmee de effectiviteit van de reis wordt geëvalueerd en hoe de resultaten worden verantwoord richting de opleiding en overige stakeholders.

\subsection{Evaluatiemethoden}

De evaluatie van de studiereis zal plaatsvinden met behulp van zowel kwantitatieve als kwalitatieve methoden:

\begin{itemize}
	\item \textbf{Enquêtes onder studenten:} direct na de reis ontvangen alle deelnemers een digitale vragenlijst (via tools zoals Google Forms of Mentimeter) waarin zij hun ervaringen, leeropbrengsten en verbeterpunten kunnen aangeven. Hierbij wordt gebruikgemaakt van Likert-schalen om tevredenheid meetbaar te maken, aangevuld met open vragen voor suggesties en tips.
	\item \textbf{Groepsdiscussie:} tijdens een gezamenlijke nabespreking wordt ruimte geboden voor reflectie en het delen van ervaringen. Dit biedt de mogelijkheid om context en nuance te verkrijgen naast de kwantitatieve resultaten.
	\item \textbf{Observaties begeleiders:} de docenten en begeleiders houden tijdens de reis aantekeningen bij over groepsdynamiek, betrokkenheid en organisatorische aandachtspunten.
\end{itemize}

\subsection{Verantwoordingsproces}

De resultaten van de evaluatie worden verwerkt in een \textbf{eindrapport}, waarin onder meer wordt opgenomen:
\begin{itemize}
	\item Een overzicht van de behaalde doelstellingen (SMART getoetst).
	\item De belangrijkste bevindingen uit de enquêtes en groepsdiscussie.
	\item Een vergelijking tussen de geplande doelstellingen en de feitelijke uitvoering.
	\item Aanbevelingen voor toekomstige studiereizen.
\end{itemize}

Dit rapport wordt gedeeld met:
\begin{itemize}
	\item De opleiding Technische Bedrijfskunde (als verantwoordelijke instantie).
	\item De begeleidende docenten.
	\item De activiteitencommissie en de studenten zelf.
\end{itemize}

\subsection{Theoretische onderbouwing}

Volgens de kwaliteitscirkel van Deming (\textit{Plan-Do-Check-Act}) \cite{deming1986out} is de fase van evaluatie en verantwoording cruciaal om continu te verbeteren. Door structureel feedback te verzamelen en terug te koppelen, wordt niet alleen de kwaliteit van toekomstige studiereizen verhoogd, maar ook de verantwoording naar stakeholders geborgd.

\subsection{Tijdsplanning evaluatie}

\begin{itemize}
	\item \textbf{Tijdens de reis:} observaties en korte polls over de organisatie en inhoud van het programma.
	\item \textbf{Direct na terugkomst:} digitale enquête en groepsreflectie.
	\item \textbf{Binnen twee weken na terugkomst:} oplevering van het evaluatierapport en presentatie aan de opleiding.
\end{itemize}