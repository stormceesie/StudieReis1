\section{Doelstelling en scope}

Dit hoofdstuk definieert de doelstelling en scope van de buitenland reis.

\subsection{Doelstelling}

De internationale studiereis van BK5, opleiding Technische Bedrijfskunde aan Saxion University, heeft als doel studenten en ervaring te laten opdoen in een internationale context. Door het bezoeken van bedrijven in Boedapest of nabijgelegen plaatsen maken studenten kennis met buitenlandse bedrijfsprocessen, managementstructuren en culturele verschillen in ondernemerschap. Naast educatieve waarde wordt met de studiereis ook beoogd de groepsverbinding te versterken door gezamenlijke activiteiten te organiseren die bijdragen aan samenwerking en interculturele vaardigheden.

\vspace{0.5cm}

De studiereis draagt bij aan:

\begin{enumerate}
	\item Het vergroten van de internationale oriëntatie van studenten.
	\item Het opdoen van praktijkervaring door bedrijfsbezoeken en prestaties.
	\item Het stimuleren van samenwerking en teambuilding binnen de klas.
	\item Het verrijken van de studie door het koppelen van theorie aan praktijk in een buitenlandse omgeving.
\end{enumerate}

\subsection{Scope}

De scope van dit projectplan richt zich op de voorbereiding, organisatie en uitvoering van de studiereis naar Boedapest. Dit omvat:

\textbf{Binnen de scope:}

\begin{itemize}
	\item Selectie en bevestiging van de bestemming (Boedapest).
	\item Vastleggen van leerdoelen en educatieve waarde van de reis.
	\item Organisatie van bedrijfsbezoeken in overleg met lokale bedrijven en instellingen.
	\item Ontwikkeling van een programma met zowel educatieve als culturele activiteiten, in samenwerking met de activiteitencommissie.
	\item Financiële planning en begroting, inclusief afstemming met de opleiding en het opstellen van een kostenoverzicht voor studenten.
	\item Logistieke organisatie, zoals vervoer, verblijf en dagplanning.
	\item Communicatie met studenten, begeleiders en stakeholders.
\end{itemize}

\textbf{Buiten de scope:}

\begin{itemize}
	\item Individuele reisarrangementen of persoonlijke uitstapjes buiten het programma.
\end{itemize}