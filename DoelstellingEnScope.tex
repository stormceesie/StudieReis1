\section{Doelstelling en scope}

Dit hoofdstuk definieert de doelstelling en de scope van de internationale studiereis. Om de effectiviteit en haalbaarheid van het project te waarborgen, is de doelstelling volgens het SMART-criteria geformuleerd \cite{doran1981smart}.

\subsection{Doelstelling}

De internationale studiereis van BK5, opleiding Technische Bedrijfskunde aan Saxion University, heeft als doel studenten in een periode van \textbf{5 dagen} (tijdsgebonden)
internationale ervaring te laten opdoen in \textbf{Boedapest, Hongarije} (specifiek). Tijdens de reis worden minimaal \textbf{vijf bedrijfsbezoeken} georganiseerd, waarbij
studenten kennismaken met \textbf{bedrijfsprocessen, managementstructuren en culturele verschillen} in een internationale context (meetbaar). De studiereis wordt uitgevoerd
met een budget dat vooraf door de activiteitencommissie en opleiding is goedgekeurd (acceptabel en realistisch).  

\vspace{1cm}

Naast de educatieve waarde is de reis erop gericht om de \textbf{groepscohesie en interculturele samenwerking} te versterken door het organiseren
van gezamenlijke culturele en sociale activiteiten. De ambitie is dat studenten na afloop aantoonbaar beter in staat zijn om theoretische kennis
te koppelen aan praktische toepassingen in een internationale bedrijfsomgeving, en hun interculturele competenties hebben vergroot (ambitieus).

De studiereis draagt concreet bij aan:

\begin{enumerate}
	\item Het vergroten van de internationale oriëntatie van studenten.
	\item Het opdoen van praktijkervaring door bedrijfsbezoeken en prestaties.
	\item Het stimuleren van samenwerking en teambuilding binnen de klas.
	\item Het verrijken van de studie door het koppelen van theorie aan praktijk in een buitenlandse omgeving.
\end{enumerate}

\subsection{Scope}

De scope van dit projectplan richt zich op de voorbereiding, organisatie en uitvoering van de studiereis naar Boedapest. Dit omvat:

\textbf{Binnen de scope:}

\begin{itemize}
	\item Selectie en bevestiging van de bestemming (Boedapest).
	\item Vastleggen van leerdoelen en educatieve waarde van de reis.
	\item Organisatie van bedrijfsbezoeken in overleg met lokale bedrijven en instellingen.
	\item Ontwikkeling van een programma met zowel educatieve als culturele activiteiten.
	\item Financiële planning en begroting, inclusief afstemming met de opleiding en het opstellen van een kostenoverzicht voor studenten.
	\item Logistieke organisatie, zoals vervoer, verblijf en dagplanning.
	\item Communicatie met studenten, begeleiders en stakeholders.
\end{itemize}

\textbf{Buiten de scope:}

\begin{itemize}
	\item Individuele reisarrangementen of persoonlijke uitstapjes buiten het programma.
\end{itemize}

\noindent Het expliciet afbakenen van de scope voorkomt misverstanden en draagt bij aan een efficiënte uitvoering van het project \cite{pmibook2021}.