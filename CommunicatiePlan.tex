\section{Communicatie plan}
Het communicatieplan beschrijft hoe wij als projectgroep communiceren met alle betrokken partijen rondom de studiereis. Het doel is dat alle informatie duidelijk, tijdig en gestructureerd wordt gedeeld, zodat de reis soepel verloopt en iedereen goed geïnformeerd is.

\subsection{Communicatie met studenten}
Met de studenten wordt voornamelijk gecommuniceerd over de inhoud van de reis, de dagelijkse planning en praktische zaken. Persoonlijk contact in de klas wordt gebruikt voor aankondigingen en uitleg. Daarnaast worden er digitale flyers via de e-mail verstuurd met belangrijke informatie zoals de bedrijfsbezoeken, tijdsindelingen en locaties. Voorafgaand aan de reis ontvangen de studenten een e-mail met onder andere de kamerindeling, een paklijst en andere praktische informatie. Tijdens de reis kan een whatsapp groep worden gebruikt voor korte vragen en updates.

\subsection{Communicatie met andere commissies en het bestuur}
Voor de afstemming met andere commissies wordt om de twee weken een vergadering georganiseerd met alle voorzitters van de commissies. Deze vergaderingen dienen om de voortgang te bespreken, belangrijke besluiten af te stemmen en om te zorgen dat alle commissies goed samenwerken. Het bestuur en dus de penningmeester is ook bij deze vergaderingen aanwezig, wat belangrijk is om de kosten goed in de gaten te houden. Daarnaast kan e-mail worden gebruikt om extra informatie te delen.

\vspace{6cm}

\subsection{Communicatie met bedrijven}
De communicatie met bedrijven waar de bedrijfsbezoeken zullen plaats vinden verloopt voornamelijk via e-mail en telefonisch contact. Hiervoor wordt een nieuw e-mailaccount aangemaakt zodat al het contact op één plek terug te vinden is. Voor directe afspraken of dringende vragen kan er ook telefonisch contact zijn.

\subsection{Communicatie met docenten}
Wij houden de docenten op de hoogte van de voortgang van de studiereis en belangrijke beslissingen. Er kan gecommuniceerd worden via e-mail, in de lessen en tijdens vergaderingen voor toelichting en feedback. Zo wordt ervoor gezorgd dat docenten betrokken zijn en waar nodig ondersteuning kunnen bieden.

\subsection{Onderlinge communicatie binnen de projectgroep}
Binnen de projectgroep wordt samengewerkt met behulp van gezamenlijke documenten voor documentatie, planning en taakverdeling. Op school wordt er overlegt om de voortgang te bespreken, taken te verdelen en beslissingen te nemen. Voor korte vragen, updates en korte communicatie wordt Teams of whatsapp gebruikt.