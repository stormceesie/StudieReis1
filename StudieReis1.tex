% LaTeX template for BK5 documents IEEE based
\documentclass{BK5}

\usepackage{amsmath}
\usepackage{amssymb}
\usepackage{cite}
\usepackage[utf8]{inputenc}
\usepackage{hyperref} % To be able to click on references etc
\usepackage{lipsum}
\usepackage{multicol}
\usepackage{graphicx}
\usepackage{tmi}
\usepackage{float}

\author{Stijn van Straten {\small (493809)}\\
	Marijn van der Gracht {\small (499194)}\\
	Tristan van Duuren {\small (480101)}\\
	Florent Kegler {\small (514277)}
}
\date{\today}

\graphicspath{{Images/}}

% Metadata
\BKproject{Internationale studiereis BK5 1}
\BKthesisSubTitle{Een internationale ervaring voor technische bedrijfskunde studenten}
\BKuniversity{Saxion University}
\BKaddress{M. H. Tromplaan 28, Enschede, Netherlands}
\BKCompanyCoach{drs. Hans R. Niekus} % Most metadata are optional

\BKContact{%
	\hspace*{0em}514277@student.saxion.nl\\
	\hspace*{4em}499194@student.saxion.nl\\
	\hspace*{4em}480101@student.saxion.nl\\
	\hspace*{4em}493809@student.saxion.nl
}

% AI note on the bottom of the file
%\renewcommand{\BKAINote}{Different AI note}

\begin{document}
	\maketitle
	
	\begin{abstract}
		losstaande samenvatting
	\end{abstract}
	
	\section{Over ons}
	
	\begin{minipage}{0.5\linewidth}
		\includegraphics[width=\linewidth]{unknown.jpg}
	\end{minipage}
	\hfill
	\vspace{1cm}
	\begin{minipage}{\linewidth}
		\textbf{Florent Kegler (514277)} \\
		\textbf{Rol:} Bedrijfscontact \\
		Ik ben 21 jaar en woon mijn hele leven in Rijssen. Ik heb de opleiding Mechatronica gevolgd aan Saxion University in Enschede. Binnen dit project neem ik de rol van bedrijfscontact op mij. Dat betekent dat ik verantwoordelijk ben voor het onderhouden van het contact met de bedrijven die we tijdens onze studiereis in het buitenland zullen bezoeken.
	\end{minipage}
	
	\vspace{1cm}
	
	\begin{minipage}{0.5\linewidth}
		\includegraphics[width=\linewidth]{unknown.jpg}
	\end{minipage}
	\hfill
	\vspace{1cm}
	\begin{minipage}{\linewidth}
		\textbf{Tristan van Duuren (480101)} \\
		\textbf{Rol:}  \\
		Stukje over jezelf
	\end{minipage}
	
	\vspace{1cm}
	\newpage
	
	\begin{minipage}{0.5\linewidth}
		\includegraphics[width=\linewidth]{unknown.jpg}
	\end{minipage}
	\hfill
	\vspace{1cm}
	\begin{minipage}{\linewidth}
		\textbf{Stijn van Straten (493809)} \\
		\textbf{Rol:}  \\
		Stukje over jezelf
	\end{minipage}
	
	\vspace{1cm}
	
	\begin{minipage}{0.5\linewidth}
		\includegraphics[width=\linewidth]{unknown.jpg}
	\end{minipage}
	\hfill
	\vspace{1cm}
	\begin{minipage}{\linewidth}
		\textbf{Marijn van der Gracht (499194)} \\
		\textbf{Rol:} Groepsleider \\
		Stukje over jezelf
	\end{minipage}
	
	\newpage
	
	\tableofcontents
	
	\section{Inleiding}
	\IEEEPARstart{D}{it} document vormt het projectplan voor de internationale studiereis van BK5, een klas van de opleiding Technische Bedrijfskunde aan Saxion University. Het doel van dit plan is om de voorbereiding en uitvoering van de studiereis gestructureerd vast te leggen en te onderbouwen.
	
	In dit plan worden onder andere de doelstellingen van de studiereis beschreven, evenals de keuzes die gemaakt worden ten aanzien van de bestemming en de bijbehorende kosten. Ook worden de stakeholders in kaart gebracht en de risico’s die verbonden zijn aan de organisatie en uitvoering van de reis besproken.
	
	Daarnaast bevat het projectplan een planning van de belangrijkste activiteiten en een overzicht van de bedrijven en organisaties die tijdens de reis bezocht zullen worden. Hiermee wordt niet alleen de leerwaarde van de reis gewaarborgd, maar ook het organisatorische kader waarin de studiereis plaatsvindt.
	
	Het projectplan dient als leidraad voor de voorbereiding, uitvoering en evaluatie van de studiereis, en vormt de basis voor verantwoording richting begeleiders, medestudenten en betrokkenen.
	
	\section{Doelstelling en scope}
	
	\section{Bestemming en programma}
	
	\section{Stakeholders}
	
	Voor de organisatie van de internationale studiereis zijn verschillende partijen betrokken. Onderstaand overzicht geeft een stakeholderanalyse weer.
	
	\begin{table*}[h!]
		\centering
		\caption{Stakeholderanalyse studiereis BK5}
		\label{tab:stakeholders}
		\begin{tabular}{|l|p{4.5cm}|l|p{4.5cm}|}
			\hline
			\textbf{Stakeholder} & \textbf{Rol / Belang} & \textbf{Invloed / Macht} & \textbf{Communicatie} \\
			\hline
			Projectgroep BK5 (wij) & Organiseren van de studiereis, opstellen planning en budget, contact met bedrijven & Hoog & Regelmatige overleggen, e-mail, projectmeetings \\
			\hline
			Studenten BK5 & Deelnemers van de studiereis, input over voorkeuren en wensen & Gemiddeld & Presentaties, enquêtes, groepsapps \\
			\hline
			Docenten / Begeleiders & Begeleiden van de studiereis, goedkeuring van planning en budget & Hoog & Overleggen, e-mail, voortgangsrapportages \\
			\hline
			Bedrijven & Bezoeken en samenwerking, mogelijk gastcolleges of rondleidingen & Laag tot gemiddeld & E-mailcontact, afspraken voorafgaand aan bezoek, telefonische follow-up \\
			\hline
			Saxion University / School BBT & Formele goedkeuring, verzekeringen, veiligheid & Hoog & Officiële correspondentie, voortgangsrapportages \\
			\hline
		\end{tabular}
		
	\end{table*}
	
	\section{Planning en aanpak}
	
	\section{Budget en financiën}
	
	\section{Risicoanalyse en beheersmaatregelen}
	
	\section{Communicatieplan}
	
	\section{Evaluatie en verantwoording}
	
	\section{Conclusie}
	
\end{document}