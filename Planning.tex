\section{Planning}

Om de buitenlandse studiereis tot een groot succes te maken, is een goede voorbereiding essentieel. In dit hoofdstuk is te lezen waaruit de planning opgebouwd is en wanneer onderdelen klaar moeten zijn.

\vspace{7cm}

\begin{table}[h!]
	\centering
	\caption{Voorbereiding en Organisatie}
	\label{tab:voorbereiding}
	\begin{tabular}{|p{4cm}|p{6cm}|p{5cm}|}
		\hline
		\textbf{Activiteiten} & \textbf{Omschrijving} & \textbf{Deadline} \\
		\hline
		Taakverdeling commissie & Benoemen van voorzitter, penningmeester, bedrijfscontactpersoon en activiteitenverantwoordelijke & 7 september \\
		\hline
		Keuze bestemming & Definitieve keuze maken voor de bestemming & 14 september \\
		\hline
		Inschrijving & Start van de inschrijving voor studenten voor de buitenlandse reizen & 21 september \\
		\hline
		Enquête activiteiten & Peilen van studenteninteresses via een enquête & 28 september \\
		\hline
		Projectplan inleveren & Indienen van het conceptplan bij Hans voor feedback & 28 september \\
		\hline
	\end{tabular}
\end{table}

\begin{table}[h!]
	\centering
	\caption{Logistiek en Bedrijven}
	\label{tab:logistiek}
	\begin{tabular}{|p{4cm}|p{6cm}|p{5cm}|}
		\hline
		\textbf{Activiteiten} & \textbf{Omschrijving} & \textbf{Deadline} \\
		\hline
		Vluchten en accommodatie boeken & Boeken van vliegtickets en verblijf & 28 september \\
		\hline
		Contact opnemen bedrijven & Bedrijven benaderen voor excursies & 19 oktober \\
		\hline
		Begroting presenteren & Conceptbegroting voorleggen aan het bestuur & 19 oktober \\
		\hline
		Bedrijfsbezoeken definitief & Minimaal 3-5 bedrijfsbezoeken vastleggen & 16 december \\
		\hline
	\end{tabular}
\end{table}

\begin{table}[h!]
	\centering
	\caption{Activiteiten en Programma}
	\label{tab:activiteiten}
	\begin{tabular}{|p{4cm}|p{6cm}|p{5cm}|}
		\hline
		\textbf{Activiteiten} & \textbf{Omschrijving} & \textbf{Deadline} \\
		\hline
		Activiteiten definitief kiezen & Activiteiten selecteren op basis van studenteninteresses & 16 december \\
		\hline
		Vlucht boeken & Bevestigen en boeken van vlucht en verblijf & 16 december \\
		\hline
		Programma maken & Uitwerken van een conceptprogramma & 16 december \\
		\hline
		Gezamenlijk diner organiseren & Restaurant kiezen en reserveren voor gezamenlijk diner & 16 december \\
		\hline
		Cultuuractiviteiten kiezen & Selecteren van culturele activiteiten passend bij studenteninteresses & 16 december \\
		\hline
	\end{tabular}
\end{table}

\newpage

\subsection{Opbouw}

Om de studiereis zo soepel mogelijk te laten verlopen, word vooraf een duidelijk aanpak opgesteld. Hierdoor is het voor alle betrokken studenten helder wat er van hen verwacht word en op welk moment er bepaalde taken uitgevoerd moeten worden. Verder staat er ook aangegeven wanneer er taken definitief vastgezet moeten worden. Elk commissielid heeft een eigen verantwoordelijkheid. Door de planning aan te houden wordt er gegarandeerd dat de voorbereiding in goede banen verloopt. 

De deadlines in de voorbereidende fase zijn bewust vroeg in het collegejaar vastgesteld, zodat er al snel grote stappen konden worden gezet. Het tijdig boeken van vluchten en accommodaties levert namelijk aanzienlijke kostenbesparingen op. Daarnaast biedt het vroeg benaderen van bedrijven voldoende ruimte om eventuele afwijzingen op te vangen. Voor de planningsfase werd een deadline bepaald in de laatste week voor de kerstvakantie, terwijl voor de eindfase een deadline geldt in de week voorafgaand aan de studiereis.